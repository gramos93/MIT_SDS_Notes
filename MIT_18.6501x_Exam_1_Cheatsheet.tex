\documentclass[9pt,landscape]{article}
\usepackage{amssymb,amsmath,amsthm,amsfonts}
\usepackage{multicol,multirow}
\usepackage{calc}
\usepackage{ifthen}
\usepackage[landscape]{geometry}
\usepackage[colorlinks=true,citecolor=blue,linkcolor=blue]{hyperref}
\usepackage{enumitem}


\ifthenelse{\lengthtest { \paperwidth = 11in}}
    { \geometry{top=.5in,left=.5in,right=.5in,bottom=.5in} }
	{\ifthenelse{ \lengthtest{ \paperwidth = 297mm}}
		{\geometry{top=1cm,left=1cm,right=1cm,bottom=1cm} }
		{\geometry{top=1cm,left=1cm,right=1cm,bottom=1cm} }
	}
\pagestyle{empty}
\makeatletter
\renewcommand{\section}{\@startsection{section}{1}{0mm}%
                                {-1ex plus -.5ex minus -.2ex}%
                                {0.5ex plus .2ex}%x
                                {\normalfont\large\bfseries}}
\renewcommand{\subsection}{\@startsection{subsection}{2}{0mm}%
                                {-1explus -.5ex minus -.2ex}%
                                {0.5ex plus .2ex}%
                                {\normalfont\normalsize\bfseries}}
\renewcommand{\subsubsection}{\@startsection{subsubsection}{3}{0mm}%
                                {-1ex plus -.5ex minus -.2ex}%
                                {1ex plus .2ex}%
                                {\normalfont\small\bfseries}}
\makeatother
\setcounter{secnumdepth}{0}
\setlength{\parindent}{0pt}
\setlength{\parskip}{0pt plus 0.5ex}
% -----------------------------------------------------------------------

\title{MIT 18.650x Mid-Term 1 Cheatsheet}

\begin{document}

\raggedright
\footnotesize

\begin{center}
     \Large{\textbf{MIT 18.650x Mid-Term 1 Cheatsheet}} \\
\end{center}
\begin{multicols}{3}
\setlength{\premulticols}{1pt}
\setlength{\postmulticols}{1pt}
\setlength{\multicolsep}{1pt}
\setlength{\columnsep}{1pt}

\section{Unit 2 - Parametric Estimation and Confidence Intervals}
In this unit, we introduce a mathematical formalization of statistical modeling to make a principled sense of the \underline{Trinity of Statistical} inference.

\begin{enumerate}
    \item {Estimation: $\hat{p} = \overline{R_n}$ is an estimator for the real parameter $p^*$.}
    \item {Confidence intervals: $[0.56, 0.73]$ is a 95\% confidence interval for $p$.}
    \item {Hypothesis testing: "We found \textit{statistical evidence} that more couples turn their head to the right."}
\end{enumerate}

\subsection{Chapter 4 - Inequalities}
Inequalities are used to bound quantities that might otherwise be hard to compute. \\

\textbf{Markov's inequality:} let $X$ be a non-negative r.v. and suppose $\mathbb{E}(X)$ exists. For any $t > 0$,
$$\mathbb{P}(X > t) = \frac{\mathbb{E}[X]}{t}$$

\textbf{Chebyshev's inequality:} let $\mu = \mathbb{E}[X]$ and $\sigma^2 = Var(X)$. Then, $$\mathbb{P}(|X - \mu| \ge t) \leq \frac{\sigma^2}{t^2}$$

\textbf{Hoeffding's inequality:} let $X_1, \dots, X_n \sim{} Ber(p)$ then for any $\epsilon > 0$, $$\mathbb{P}(|\overline{X_n} - p| > \epsilon) \leq 2e^{\frac{-2n\epsilon^2}{(b-a)^2}}$$
Where $\overline{X_n}$ is the sample average and $X \in [a, b]$.\\

\textbf{Jensen's inequality:} If $g$ is convex, then $$\mathbb{E}[g(X)] \ge g(\mathbb{E}[X])$$
The reverse is true if $g$ is concave.

\subsection{Chapter 5 - Convergence of r.v.'s}
There are 2 main ideas in this chapter. First, the \textbf{LLN} says that the \textit{sample average} converges in probability to $\mathbb{E}$. Second, the \textbf{CTL} says that
$\sqrt{n}(\overline{X_n} - \mu)$ converges in distribution to a Normal distribution for large enough $n$.\\
Let $X_1, \dots, X_n$ be a sequence of r.v's, $X$ is another r.v. Let $F_n(t)$ denote the CDF of $X_n$ an $F(t)$ denote the CDF of $X$.\\

\textbf{Convergence in probability:} $X_n \overset{P}{\to} X$ if, for every $\epsilon > 0$, $$\mathbb{P}(|X_n - X| > \epsilon) \to 0$$ as $n \to \infty$. 
\textbf{Warning !} One might think that $X_n \overset{P}{\to} b$ implies $\mathbb{E}[X] \to b$. This is \textbf{not} true. 

\textbf{Convergence in distribution:} $X_n \overset{d}{\to} X$ if, $$\lim_{n \to \infty} F_n(t) = F(t)$$ for all $t$ for which $F$ is continuous. 

\subsubsection{Properties of convergence}
The following relationships hold:
\begin{enumerate}[itemsep=0em]
    \item{$X_n \overset{P}{\to} X$ implies that $X_n \overset{d}{\to} X$.}
    \item{If $X_n \overset{d}{\to} X$ and if $\mathbb{P}(X=c) = 1$ for some real number $c$, then $X_n \overset{P}{\to} X$.}
    \item{If $X_n \overset{P}{\to} X$ and $Y_n \overset{P}{\to} Y$ then, $X_n + Y_n \overset{P}{\to} X+Y$.}
    \item{If $X_n \overset{d}{\to} X$ and $Y_n \overset{d}{\to} c$ then, $X_n + Y_n \overset{c}{\to} X + c$.}
    \item{If $X_n \overset{P}{\to} X$ and $Y_n \overset{P}{\to} Y$ then, $X_n Y_n \overset{P}{\to} XY$.}
    \item{If $X_n \overset{d}{\to} X$ and $Y_n \overset{d}{\to} c$ then, $X_n Y_n \overset{d}{\to} cX$.}
    \item{If $X_n \overset{P}{\to} X$ then, $g(X_n) \overset{P}{\to} g(X)$.}
    \item{If $X_n \overset{d}{\to} X$ then, $g(X_n) \overset{d}{\to} g(X)$.}
\end{enumerate}

Parts (4) and (6) are known as the \textbf{Slutzky's theorem}. Note that $X_n \overset{d}{\to} X$ and $Y_n \overset{d}{\to} Y$ does not imply that $X_n = Y_n \overset{d}{\to} X+Y$.

\subsubsection{Weak LLN \& CTL}
\textbf{Interpretation of the WLLN:}The distribution of $\overline{X_n}$ becomes more concentrated around $\mu$ as $n$ get larger. From the Chebyshev's inequality: 
$\mathbb{P}(|\overline{X_n} - \mu| > \epsilon) \leq \frac{Var(\overline{X_n})}{\epsilon^2} = \frac{\sigma^2}{n \epsilon^2}.$

\textbf{Central Limit Theorem:}let $X_1, \dots, X_n$ be i.i.d with mean $\mu$ and variance $\sigma^2$. Then 
$$Z_n \equiv \frac{\overline{X_n} - \mu}{\sqrt{Var(\overline{X_n})}} = \frac{\sqrt{n} (\overline{X_n} - \mu)}{\sigma} \overset{(d)}{\to} Z$$
where $Z$ is a standard Normal.\\
\textbf{Multivariate} version of the CTL. Let $X_1, \dots, X_n$ be i.i.d random \textit{vectors}, $X_i = [X_{1}, X_{2}, \dots, X_{k}]^T$ with mean $\mu = [\mu_{1}, \mu_{2}, \dots, \mu_{k}]^T$
and variance matrix $\Sigma$. Then,$$\sqrt{n}(\overline{X} - \mu) \overset{(d)}{\to} N(0, \Sigma)$$

\subsubsection{The Delta Method}
The Delta Method allows us to find the limiting distribution of $g(Y_n)$ where $g$ is a \underline{smooth} function.
Suppose that $\frac{\sqrt{n}(Y_n - \mu)}{\sigma} \overset{(d)}{\to} N(0, 1)$ and $g$ is a \underline{differentiable} function such that $g'(\mu) \ne 0$. Then,

$$\sqrt{n}(g(Y_n) - g(\mu)) \overset{(d)}{\to} N(0, g'(\mu)|^2 \sigma^2)$$

\textbf{Multivariate} version of the Delta method. Let $Y_n = (Y_{n1}, \dots, Y_{nk})$ be a sequence random \textit{vectors} such that $\sqrt{n}(Y_n - \mu) \overset{(d)}{\to} N(0, \Sigma)$.
In this case $g: \mathbb{R}^k \to \mathbb{R}.$ Then,

$$\sqrt{n}(g(Y_n) - g(\mu)) \overset{(d)}{\to} N(0, \nabla g(\mu)^T \Sigma \nabla g(\mu))$$


\subsection{Chapter 6 Statistical Inference}
\textit{"\dots is the process of using data to infer the distribution that generated that data."}\\

\subsubsection{Point estimation: } 

Provides a \textit{best guess} of some quantity of interest. The point estimate of $\theta$, a fixed number, is $\hat{\theta}$, which depends on the data, therefore a r.v. A point estimator $\hat{\theta}_n$ of the parameter $\theta$ is \textbf{consistent} if $\hat{\theta}_n \overset{P}{\to} \theta$.
The \textbf{bias} of an estimator is defined as $bias(\hat{\theta}_n) = \mathbb{E_{\theta}}[\hat{\theta}_n] - \theta$. An estimator is unbiased if $\mathbb{E_{\theta}}[\hat{\theta}_n] = 0$.
Following the bias, the \textbf{mean squared error} (MSE) is $MSE = bias^2(\hat{\theta}_n) + \mathbb{V_{\theta}}(\hat{\theta}_n)$.\\

\subsubsection{Confidence Intervals: } 
Is an interval $I = (a, b)$ such that $\mathbb{P}_{\theta}(\theta \in I) \ge 1-\alpha$, for all $\theta \in \Theta$. 
$I$ is random, but $\theta$ is not random. Suppose that $\widehat{\theta}_n \approx N(\theta, \sigma^2)$. The confidence interval will be of the form 
$$I = \left [\widehat{\theta}_n - q_{\alpha/2} \frac{\sqrt{\sigma^2}}{\sqrt{n}}, \widehat{\theta}_n + q_{\alpha/2} \frac{\sqrt{\sigma^2}}{\sqrt{n}} \right ]$$

However, most of the time, this confidence interval will depend on the true parameter $\theta$. There are 3 methods to fix this problem.\\

\textbf{Conservative Bound:} Suppose we're calculating a confidence interval for the parameter $p$ of a Bernoulli r.v. where $\sigma = \sqrt{p(1-p)}$. 
In this case $p(1-p)$ has the \textit{upper bound} $1/4$. We can replace this value in $I$. 
Then $I = [\widehat{p} - q_{\alpha/2} \frac{1}{2\sqrt{n}}, \widehat{p} + q_{\alpha/2} \frac{1}{2\sqrt{n}}]$.\\

\textbf{Plug-in:} Since $\widehat{\theta}$ is a consistent estimator of $\theta$, we can simply replace $\theta$ with $\widehat{\theta}$.

\textbf{Solve the equation:} A more elaborate method is to solve the system of two Inequalities w.r.t $\theta$. 
$$\widehat{\theta} - q_{\alpha/2} \frac{\sqrt{\sigma^2}}{\sqrt{n}} \leq \theta \leq \widehat{\theta} + q_{\alpha/2} \frac{\sqrt{\sigma^2}}{\sqrt{n}}$$
$$(\theta - \widehat{\theta})^2 \leq q^2_{\alpha/2} \frac{\sigma^2}{n}$$
Finding the roots of the quadratic equation will lead us to $I_{solve} = (\theta_1, \theta_2)$.

\subsection{Hypothesis Testing and p--value}


\end{multicols}

\end{document}

